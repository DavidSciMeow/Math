\documentclass[11pt]{article}

% XeLaTeX 字体与中文支持(使用 XeLaTeX 编译)
\usepackage{fontspec}
\usepackage{xeCJK}
\setmainfont{Times New Roman}
\setCJKmainfont[ItalicFont={SimSun},AutoFakeSlant=0.2,BoldFont={SimSun}]{SimSun}
%\newcommand{\songti}{\CJKfamily{zhsong}}

% 数学包
\usepackage{amsmath,amssymb}

% 定理环境(提供 \theoremstyle)
\usepackage{amsthm}
\newtheorem{theorem}{定理}
\newtheorem{proposition}{命题}
\newtheorem{corollary}{推论}
\newtheorem{example}{例}

% 可选:超链接与微排版
\usepackage{hyperref}
\usepackage{microtype}
\usepackage[a4paper,margin=1.5cm]{geometry}

\begin{document}

\title{本仓库的数学命题证明:有理数的整数放大性质及其扩充}
\author{DavidSciMeow}
\date{Oct, 24, 2025}
\maketitle

\section{记号与补充约定}
\begin{itemize}
  \item $\mathbb{Q}$ 表示有理数集,$\mathbb{Z}$ 表示整数集,$\mathbb{Z}_{+}$ 表示正整数集,$\mathbb{N}_0=\{0,1,2,\dots\}$。
  \item 若 $n$ 为正整数,则写其素因子分解为 $n=\prod_{\pi}\pi^{e_\pi(n)}$,其中 $\pi$ 遍历素数,$e_\pi(n)\in\mathbb{Z}_{\ge 0}$ 为素数 $\pi$ 在 $n$ 中的指数。
  \item 对素数 $\pi$ 定义 $\pi$-adic 指数对整数 $n>0$ 令 $v_\pi(n):=e_\pi(n)$。对任意有理数 $r=\dfrac{a}{b}$(互素表示,$a\in\mathbb{Z},b\in\mathbb{Z}_{+}$),可推广定义
  \[
    v_\pi(r):=v_\pi(a)-v_\pi(b).
  \]
  \item 对固定的进制基数 $p\ge2$(不必为素数),我们关心是否存在 $k\in\mathbb{N}_0$ 使得 $p^k q\in\mathbb{Z}$。
\end{itemize}

\section{基本命题与证明(有理数情形)}

\begin{theorem}
  对任意 $q\in\mathbb{Q}$,存在正整数 $K\in\mathbb{Z}_{+}$ 使得 $Kq\in\mathbb{Z}$。
\end{theorem}

\begin{proof}
任取 $q\in\mathbb{Q}$,将其表示为最简分数
\[
q=\frac{a}{b},
\]
其中 $a\in\mathbb{Z},\ b\in\mathbb{Z}_{+}$,且 $\gcd(a,b)=1$。取 $K=b$,则
\[
Kq=b\cdot\frac{a}{b}=a\in\mathbb{Z}.
\]
因此存在正整数 $K$(例如分母 $b$)使得 $Kq\in\mathbb{Z}$,证明完毕。
\end{proof}

\begin{theorem}
关于基 $p$ 的等价条件。设固定基数 $p\ge2$(不必为素数)。
令 $q=\dfrac{a}{b}$ 为最简分数,$a\in\mathbb{Z},\ b\in\mathbb{Z}_{+},\ \gcd(a,b)=1$。
下列命题等价:
\begin{enumerate}
  \item 存在 $k\in\mathbb{N}_0$ 使得 $p^k q\in\mathbb{Z}$。
  \item 存在 $k\in\mathbb{N}_0$ 使得 $b\mid p^k$。
  \item $b$ 的所有素因子都出现在 $p$ 的素因子分解中:对任意素数 $\pi$,若 $v_\pi(b)>0$ 则 $v_\pi(p)>0$。
\end{enumerate}
特别地,当 $p$ 为素数时,上述等价于 $b$ 為 $p$ 的幂,即 $b=p^m$ 对某个 $m\in\mathbb{N}_0$。
\end{theorem}

\begin{proof}
(1)$\Rightarrow$(2): 若存在 $k$ 使 $p^k q\in\mathbb{Z}$,则 $p^k\cdot\frac{a}{b}\in\mathbb{Z}$,即 $b\mid p^k a$。由于 $\gcd(a,b)=1$,故 $b\mid p^k$。

(2)$\Rightarrow$(1): 若 $b\mid p^k$,则 $p^k q=p^k\cdot\frac{a}{b}\in\mathbb{Z}$。

(2)$\Leftrightarrow$(3): 将 $p$ 与 $b$ 写成素因子分解
\[
p=\prod_{\pi}\pi^{v_\pi(p)},\qquad b=\prod_{\pi}\pi^{v_\pi(b)}.
\]
条件 $b\mid p^k$ 等价于对每个素数 $\pi$ 有
\[
v_\pi(b)\le v_\pi(p^k)=k\,v_\pi(p).
\]
若存在某素数 $\pi$ 使 $v_\pi(b)>0$ 且 $v_\pi(p)=0$,则右侧恒为 $0$,不可能 $\ge v_\pi(b)>0$,故必须有若 $v_\pi(b)>0$ 则 $v_\pi(p)>0$。反过来若对所有出现在 $b$ 的素因子 $\pi$ 都有 $v_\pi(p)\ge1$,则取
\[
k\ge \max_{\pi:\,v_\pi(b)>0}\left\lceil\frac{v_\pi(b)}{v_\pi(p)}\right\rceil,
\]
即可保证 $k\,v_\pi(p)\ge v_\pi(b)$ 对所有 $\pi$ 成立,从而 $b\mid p^k$。
\end{proof}

\section{关于最小的的显式公式}
设 $p$ 的素因子集合为 $\{\pi_1,\dots,\pi_r\}$,且 $v_{\pi_i}(p)=f_i\ge1$。把 $b$ 在这些素因子下的指数记作 $e_i:=v_{\pi_i}(b)$(若某 $\pi_i$ 未出现在 $b$ 中则 $e_i=0$)。若存在 $\pi\notin\{\pi_1,\dots,\pi_r\}$ 使 $v_\pi(b)>0$,则不存在 $k$ 使 $b\mid p^k$。若所有出现在 $b$ 中的素因子都包含于 $\{\pi_1,\dots,\pi_r\}$,则最小的符合条件的 $k$ 为
\[
k_{\min}=\max_{1\le i\le r}\left\lceil\frac{e_i}{f_i}\right\rceil.
\]
特别地,当 $p$ 为素数时只有一项,$f_1=1$,因此 $k_{\min}=e_1=v_p(b)$,此时 $b$ 必为 $p$ 的幂。

\section{基下有限小数表示的等价性}
在基 $p$ 的位置表示中,右移小数点 $k$ 位相当于乘以 $p^k$。因此:
\[
q\ \text{在基 }p\text{ 下有有限小数表示}
\quad\Longleftrightarrow\quad
\exists k\in\mathbb{N}_0\text{ 使得 }p^k q\in\mathbb{Z}.
\]
结合前述定理可得到判定条件:$q=a/b$ 在基 $p$ 下有有限小数表示当且仅当 $b$ 的所有素因子都出现在 $p$ 的素因子分解中。对十进制($p=10$)而言,这意味着分母 $b$ 只包含素因子 $2$ 与 $5$。

\section{关于把命题推广到实数集的不可能性}
我们不能把“任意数都能被某个整数放大为整数”这一性质推广到整个实数集。更精确地,固定 $p\ge2$,定义集合
\[
S_p:=\{x\in\mathbb{R}:\exists k\in\mathbb{N}_0,\ p^k x\in\mathbb{Z}\}.
\]
显然 $S_p=\bigcup_{k\ge0} p^{-k}\mathbb{Z}$,每个 $p^{-k}\mathbb{Z}$ 与 $\mathbb{Z}$ 基数相同(可数),因此 $S_p$ 为可数并集,亦可数。由 Cantor 的定理(对角线论证)可知 $\mathbb{R}$ 是不可数的,所以 $S_p\neq\mathbb{R}$,亦即不能对所有实数都成立。

具体反例也很简单:任取无理数(如 $\sqrt{2}$),若存在非零整数 $K$ 使 $K\sqrt{2}\in\mathbb{Z}$,则 $\sqrt{2}$ 为有理数,矛盾。因此对 $\sqrt{2}$ 不存在这样的 $K$。

\subsection{Cantor 对角线论证}
Cantor 的经典证明给出 $\mathbb{R}$ 不可数:若假设存在将所有实数与自然数一一对应的枚举,那么可以构造一个与任一枚举中第 $n$ 个数在第 $n$ 位小数处不同的新实数,从而得到矛盾。该论证说明了 $\mathbb{R}$ 的势大于 $\mathbb{N}$,从而不可能包含在任何可数集合之中。这里正是用于说明 $S_p$(可数)不可能等于 $\mathbb{R}$(不可数)的根本原因。

\subsection{实数的稠密性}
补充说明:尽管 $\mathbb{Q}$ 是可数集,但在实数轴中稠密。即对任意 $x\in\mathbb{R}$ 和任意 $\varepsilon>0$,存在 $q\in\mathbb{Q}$ 使 $|x-q|<\varepsilon$。证明可用阿基米德性:取 $n\in\mathbb{N}$ 使 $1/n<\varepsilon$,令 $k=\lfloor nx\rfloor$,则 $k/n\le x<(k+1)/n$,从而 $|x-k/n|<\varepsilon$。这解释了为什么任意实数任意小邻域内都包含有理数,但并不意味着可以把所有实数都写成分母为某固定形式的有理数(或者说,都能被某个整数放大为整数)。

\newpage

\section{参考资料}
\begin{itemize}
  \item D. S. Dummit, R. M. Foote, ``Abstract Algebra''(关于整环与分式域的章节)。
  \item T. W. Hungerford, ``Algebra''(分式域与局部化的处理)。
  \item W. Rudin, ``Principles of Mathematical Analysis''(实数与有理数、稠密性的讨论)。
\end{itemize}

\section*{结论}
\begin{align*}
&\forall q\in\mathbb{Q}\ \exists K\in\mathbb{Z}_{+}:\ Kq\in\mathbb{Z}.\\
&\text{设 }q=a/b\ (\text{最简}),\ p\ge2\text{ 固定,则}\\
&\qquad\bigl(\exists k\in\mathbb{N}_0:\ p^k q\in\mathbb{Z}\bigr)
\iff
\bigl(\exists k\in\mathbb{N}_0:\ b\mid p^k\bigr)
\iff
\bigl(\forall\pi\ \text{prime},\ v_\pi(b)>0\Rightarrow v_\pi(p)>0\bigr).
\end{align*}

\bigskip

\end{document}